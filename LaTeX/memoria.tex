\documentclass[]{article}
%Busca la linea que pone /tableofcontents y empieza a escribir debajo de cada sección.
%Te explico para qué es cada cosa que sea medio relevante:
\usepackage{amsmath}
\usepackage{amssymb}
\usepackage{verbatim} %con verbatim escribes bloques de texto con letra mono.
\usepackage{graphicx} %para insertar imagenes, cuando meta yo una usa el codigo de ejemplo
\usepackage{listings}
\usepackage{fullpage}
\usepackage{color}
\usepackage{fancyvrb}
\usepackage[spanish]{babel}
\usepackage[utf8]{inputenc} %Para usar acentos directamente en latex
\usepackage{hyperref} %Para que el indice tenga hiperenlaces y si quieres poner los tuyos
\hypersetup{%
	pdfborder = {0 0 0}
}

\definecolor{mygreen}{rgb}{0,0.6,0}
\definecolor{mygray}{rgb}{0.5,0.5,0.5}
\definecolor{mymauve}{rgb}{0.58,0,0.82}

%Para insertar código: crea un recuadro con texto mono y lineas enumeradas. Puedes referenciar un fichero y no copiar y pegar aquí.
\lstset{ %
	backgroundcolor=\color{white},   % choose the background color; you must add \usepackage{color} or \usepackage{xcolor}
	basicstyle=\footnotesize,        % the size of the fonts that are used for the code
	breakatwhitespace=false,         % sets if automatic breaks should only happen at whitespace
	breaklines=true,                 % sets automatic line breaking
	captionpos=b,                    % sets the caption-position to bottom
	commentstyle=\color{mygreen},    % comment style
	frame=single,                    % adds a frame around the code
	keepspaces=true,                 % keeps spaces in text, useful for keeping indentation of code (possibly needs columns=flexible)
	numbers=left,                    % where to put the line-numbers; possible values are (none, left, right)
	numbersep=5pt,                   % how far the line-numbers are from the code
	numberstyle=\tiny\color{mygray}, % the style that is used for the line-numbers
	rulecolor=\color{black},         % if not set, the frame-color may be changed on line-breaks within not-black text (e.g. comments (green here))
	showspaces=false,                % show spaces everywhere adding particular underscores; it overrides 'showstringspaces'
	showstringspaces=false,          % underline spaces within strings only
	showtabs=false,                  % show tabs within strings adding particular underscores
	stepnumber=1,                    % the step between two line-numbers. If it's 1, each line will be numbered
	stringstyle=\color{mymauve},     % string literal style
	tabsize=4,                       
	title=\lstname                   % show the filename of files included with \lstinputlisting; also try caption instead of title
}



\title{Servicios Telemáticos Avanzados}
\author{José Luis Cánovas Sánchez\\Ezequiel Santamaría Navarro}

\begin{document}

\maketitle


\begin{abstract}
%Aquí el resumen
\end{abstract}

\tableofcontents


\section{Introducción}

Somos el grupo 3, así que nos encargaremos de las organizaciones 31 y 32.

\section{Topología}

En esta sección describimos la forma en la que se van a desplegar los servicios y como estarán
conectados entre sí. 

\subsection{Dispositivos}

Tenemos para las pruebas y el desarrollo de la práctica, las siguientes herramientas hardware:

\begin{itemize}

	\item Un \textbf{switch} con VLAN y 5 puertos.
	\item Un ordenador que actúa de \textbf{enrutador}, con tres puertos ethernet.
	\item Un \textbf{punto de acceso} WiFi para el acceso a la red.
	\item Un \textbf{iPhone} y un \textbf{portátil Mac}, útiles para probar VOIP con wifi.
	\item Dos ordenadores que actúan como \textbf{organizaciones} 31 y 32. 
	\item Dos ordenadores más para simular \textbf{clientes} haciendo peticiones.

\end{itemize}

\subsection{Topología de red}

\begin{Verbatim}
TODO: Introducir aquí un diagrama hecho con GNS3 o similar.
\end{Verbatim}

\section{Configuración de los dispositivos}

Para la configuración de los dispositivos utilizamos herramientas Makefile, y scripts escritos en bash y python, de manera que desplegar los servicios y la topología en un dispositivo esté automatizado.
\\

Usamos una estructura de directorios basada en el hardware a configurar. Es decir, en el ordenador que hace de enrutador ejecutamos un Makefile que está en el directorio ROUTER.

\section{Servicios}

\subsection{Enrutamiento}

\subsection{SNMP}

\subsubsection{Agentes}

\subsubsection{Manejador}

\subsection{Voz sobre IP}

\end{document}
